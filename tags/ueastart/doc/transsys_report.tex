% Copyright (C) 2001 Jan T. Kim <kim@inb.mu-luebeck.de>

% $Id$
% $Log$
% Revision 1.1  2005/03/08 17:12:01  jtk
% Initial revision
%
% Revision 1.5  2003/01/28 16:59:19  kim
% separated makefile for transsys_report from Makefile.am, simplified things
%
% Revision 1.4  2002/01/25 03:36:20  kim
% Wrote some paragraphs in transsys_report, hopefully removed necessity
%     of generating GL graphics after unpacking from archive
%
% Revision 1.2  2001/04/05 17:24:27  kim
% Added examples (partially finished only)
%
% Revision 1.1  2001/04/05 15:41:39  kim
% moved transsys.tex to transsys_report.tex
%
% Revision 1.1  2001/04/04 11:11:55  kim
% Initial addition of files previously not CVS managed
%

\documentclass[12pt]{article}

\usepackage[T1]{fontenc}
\usepackage{times}
\usepackage[english]{babel}
%\usepackage{cover2e}
\usepackage{alltt}
\usepackage{moreverb}
\usepackage{graphicx}


\newcommand{\transsys}{\texttt{transsys}}
\newcommand{\ltranssys}{\texttt{L-}\transsys}
\newcommand{\codeword}[1]{\texttt{#1}}
\newcommand{\amax}{\ensuremath{a_{\mathrm{max}}}}
\newcommand{\aspec}{\ensuremath{a_{\mathrm{spec}}}}
\newcommand{\factorset}{\ensuremath{\mathcal{F}}}
\newcommand{\factorsym}[1]{\ensuremath{\mathtt{#1}}}
\newcommand{\factorvar}[1]{\ensuremath{\mathit{#1}}}
\newcommand{\factorconc}[1]{\ensuremath{c_{#1}}}
\newcommand{\concset}{\ensuremath{\mathcal{C}}}
\newcommand{\geneset}{\ensuremath{\mathcal{G}}}
\newcommand{\genesym}[1]{\ensuremath{\mathtt{#1}}}
\newcommand{\genevar}[1]{\ensuremath{\mathit{#1}}}
\newcommand{\encodingset}[1]{\ensuremath{\mathcal{E}_{#1}}}
\newcommand{\decrate}[1]{\ensuremath{r_{#1}}}
\newcommand{\diffrate}[1]{\ensuremath{d_{#1}}}
\newcommand{\newterm}[1]{\textsl{#1}}

\newcommand{\keyword}[1]{\textbf{\texttt{#1}}}
\newcommand{\prgname}[1]{\texttt{#1}}
\newcommand{\cmdline}[2][\% ]{\texttt{#1\textbf{#2}}}
\newcommand{\keyboardin}[1]{\fbox{\texttt{#1}}}


\begin{document}

%\Number{unassigned}
\title{\transsys{} User Manual}
\author{Jan T.\ Kim \\[\bigskipamount]
  \texttt{kim@inb.uni-luebeck.de} \\[\bigskipamount]
  Institute for Neuro- und Bioinformatics \\
  Seelandstr.\ 1a \\
  D-23569 L\"{u}beck, Germany}
%\Authors{Jan T.\ Kim}
%\Name{kim}
%\Phone{3909-546}
%\Fax{3909-545}
%\makecover
\maketitle

\section{Introduction}

\subsection{Motivation}

During recent years, gene regulation has become a major focus in the
molecular biosciences. Many important biological processes, such as
cell differentiation, morphogenesis, diseases and responses to drug
administration, have been found to be determined by networks of
regulatory genes at the molecular level. However, there is a lack of
generic and theoretical understanding of regulatory networks. Progress
in this area would benefit from a concise and generic representation
of regulatory networks, rendering them accessible to theoretical,
statistical and modelling approaches. \transsys{}, a contraction of
"transcription factor system", is intended to be a contribution
towards this goal.


\subsection{Concept}

Regulatory gene networks are constituted by genes encoding
transcription factors and the encoded transcription factors which
regulate the expression (and hence the activity) of the regulatory
genes as well as the expression of other genes (which are sometimes
referred to as "phenotypic realizator genes"). \transsys{} attempts to
capture these two essential components and their interrelations. At
the core, \transsys{} is a formal language for specifying the core
entities constituting regulatory networks as described above. The
\transsys{} sofware package provides a set of tools for visualizing
and analyzing networks specified in the \transsys{} language.

While a concise representation of a regulatory network may already be
of some value for managing this type of biological information, a
network alone is of limited use as a model of a biological system.
Regulatory networks receive input from the biological system in which
they are embeded. These input signals can, depending on the level of
resolution or abstraction of the particular model, be thought of as
other regulatory networks, terminals of intracellular signal cascades,
intercellular signals, or environmental signals. Moreover, there are
feedback loops between the regulatory network and the pattern
formation and morphogenesis processes which unfolds in response to
activities within the network.

The major goal of \transsys{} is to capture the integration of
regulatory networks within such biological contexts, as these contexts
can be assumed to qualitatively determine the dynamics and phenomena
emerging within a network, and moreover, the biological sense made by
regulatory networks can only be captured by integratively
(holistically?) modelling a network along with its biological
context. To achieve this goal, \transsys{} is designed as a
component which conceptually can be embedded within a more extensive
modelling context. As a first case study, \transsys{} has been
combined with the Lindenmayer system formalism. The resulting model,
labelled \ltranssys{}, is described in the last part of this report.

As the modelling level of \transsys{} is the level of regulatory
networks, surrounding levels of biological organization are not
represented in much detail, e.g.\ because \transsys{} does not address
the metabolic level, no attempt is made to enforce energy or mass
conservation. The rationale for this is the observation that
transcription factor synthesis and decay accounts for a minute
fraction of the entire cellular metabolism.


\section{The \transsys{} Model}

Modelling with \transsys{} involves a two-stage process: Firstly, a
\transsys{} model has to be defined. In a \transsys{} definition, the
core elements of the regulatory network are described by a set of
numerical properties. For factors, these properties are the decay rate
and the diffusibility. The properties of genes are divided in a
promoter section, specifying which factors have activating or
repressing effects on the expression of the gene, and a product
section, specifying which factor is encoded by the gene.

Once a \transsys{} is specified, instances can be created. The
information stored in a \transsys{} instance is an array holding the
concentrations of the factors of the network. The factor and gene
properties specify a procedure for updating these concentrations,
i.e.\ for computing the factor concentrations in the next time step
given the concentrations in the current time step. In OO parlance, the
factor concentrations may be thought of as member variables of a
\transsys{} instance, while the factor and gene properties specify and
parameterize the method for simulating the internal processes taking
place within the instance during one time step.


\subsection{Factor Model}

The properties of a factor in \transsys{} are a decay rate and a
diffusion rate. The decay rate is a value between 0 and 1 that
specifies how much of the total amount of a factor decays during a
time step. The diffusion rate specifies how large a fraction of a
factor is distributed among the neighbouring \transsys{} instances.
Evidently, the existence of at least two instances that are coupled in
some way is necessary in order to get any reasonable effect from
diffusion. The details of this coupling cannot be deduced from the
\transsys{} alone, the system relies on the context model to provide
multiple instances and a connection graph of some kind. As a simple
example, these can be provided by a CA.


\subsection{Gene Model}

The gene model used in \transsys{} comprises two parts, a regulatory
block and a product block. The regulatory block describes how the
gene's activity is affected by the various factors. The product block
simply determines which factor is synthesized as the gene's product.


\section{Program Overview}

The \transsys{} package currently comprises these programs:
\begin{itemize}
\item \prgname{transcheck}: Parse \transsys{} code and write parsed
  code into output. This is used during parser developments, to ensure
  that all information from the code is read in as intended. However,
  this program may also be of some use for locating errors in
  \transsys{} code.
\item \prgname{transexpr} Generate a time series factor
  concentrations. Output file format is suitable for the
  \prgname{gnuplot} program.
\item \prgname{transps} Generate PostScript graphics of regulatory
  networks specified as \transsys{} programs
\item \prgname{transscatter} Produce scatter plots by starting off a
  \transsys{} instance with random initial factor concentrations and
  performing a given number of updates. Intended to check networks for
  robustness and to search and characterize attractors.
\item \prgname{ltrcheck} Perform some derivations of an \ltranssys{}
  program and write each \ltranssys{} string to output. Used for
  exploring, checking and debugging \ltranssys{} programs.
\item \prgname{ltransps} Produce graphical 2D rendition of
  \ltranssys{} run in PostScript
\item \prgname{ltransgl} Produce 3D rendition of \ltranssys{} run
  using OpenGL
\end{itemize}


\section{Examples}

\subsection{Writing a \transsys{} Program}

\subsubsection{The Beginning}

A \transsys{} program consists of the keyword \keyword{transsys},
followed by a name and a body which is enclosed by curly braces and
contains the actual components forming a regulatory network. As a
start, and to set up a framework, we'll start with an empty system,
i.e.\ with one that does not contain any components:
\verbatimtabinput[8]{tr_ex00.tra}


\subsubsection{Adding a Factor}

Transcription factors are specified by \keyword{factor} statements in
the body. So, let's start filling our emtpy system by adding a factor:
\verbatimtabinput[8]{tr_ex01.tra}
As you see, a factor specification has the same structure as a
\transsys{} specification: A keyword (\keyword{factor}), a name
(\factorsym{F1}) and a block enclosed by curly braces, which is empty here
but will be filled soon.


\subsubsection{Adding a Gene}

Our transcription factor \factorsym{F1} won't be much fun unless we also
provide a gene encoding it. Perhaps not surprisingly, genes are also
specified by a keyword, namely \keyword{gene}, followed by a name and
a block:
\verbatimtabinput[8]{tr_ex02.tra}


\subsubsection{Trying it out}

\begin{figure}
\centerline{
\includegraphics[width=0.45\linewidth]{tr_ex02.eps}
\includegraphics[width=0.45\linewidth]{tr_ex04.eps}
}
\caption{\label{fig_onefactor}
  Temporal dynamics of factor \factorsym{F1} with a decay rate of $1$
  (left) and a decay rate of $0.5$ (right).
}
\end{figure}

Our example \transsys{} program is not (yet) anything interesting as a
biological model or a complex system. But we can use some \transsys{}
programs on it for demonstration purposes. You can create a time
course of the concentration of \factorsym{F1} with the command
\begin{quote}
\cmdline{transexpr example.tra example.plt}
\end{quote}
The resulting file can be plotted using \prgname{gnuplot}:
\begin{quote}
\cmdline{gnuplot} \\
\cmdline[gnuplot> ]{plot 'example.plt' using 1:2 with lines}
\end{quote}
This plot, shown in Fig.\ \ref{fig_onefactor}, will show you that
the concentration of \factorsym{F1} is 1 at all time steps excepting
the initial one. This may not seem quite right to you: If 1.0 is the
amount of factor synthesized from \genesym{g1} in each time step,
should it not accumulate over time? The answer is that it could, if we
had specified a decay rate. Let's use this occasion to introduce the
\prgname{transcheck} program for revealing the default factor
properties:
\begin{quote}
\cmdline{transcheck example.tra}
\end{quote}
gives you the an output of our example \transsys{} program in which
the \factorsym{F1} block reads
\verbatimtabinput[8]{tr_ex03.tra}
As we see, each factor has two properties, a decay rate and a
diffusion rate. If no specification is given in a factor's block,
\transsys{} assumes the parameter to be 1.  A decay rate of 1 means
that all factor molecules (i.e. 100\% of them) decay within one time
step. This explains why the factor concentration does not exceed 1.


\subsubsection{Specifying a Decay Rate}

It would not make any sense if the decay rate could not be changed,
and quite obviously, this is done as seen in the \prgname{transcheck}
output shown above. So, let's specify the decay rate explicitly:
\verbatimtabinput[8]{tr_ex04.tra}
If you run \prgname{transexpr} on the \transsys{} program with the
factor definition modified this way, you'll see that \factorsym{F1}
indeed accumulates over time, approaching a limit of 2
asymptotically, see Fig.\ \ref{fig_onefactor}.


\subsubsection{Controlling a Gene's Expression}

\begin{figure}
\centerline{\includegraphics[width=0.45\textwidth]{tr_ex05.eps}}
\caption{\label{fig_twofactors}
  Dynamics of concentration of \factorsym{F1} and \factorsym{F2} in a
  \transsys{} program where \factorsym{F1} regulates expression of the
  gene encoding \factorsym{F2}.
}
\end{figure}
In the example \transsys{} program developed so far, the gene
\genesym{g1} is constitutively expressed. Now, it's time to take a
central step towards modelling regulatory networks by introducing
another gene controlled by \factorsym{F1}. We do this by adding the
following to our \transsys{} program:
\verbatimtabinput[8]{tr_ex05.tra}
You can check the resulting dynamics in factor concentrations by
running:
\begin{quote}
\cmdline{transexpr example.tra example.plt} \\
\cmdline{gnuplot} \\
\cmdline[gnuplot> ]{plot 'example.plt' using 1:2 with lines} \\
\cmdline[gnuplot> ]{replot 'example.plt' using 1:3 with lines}
\end{quote}
The results are also shown in Fig.\ \ref{fig_twofactors}. The dynamics
of \factorsym{F1} are as in the preceding versions of the example. To
understand the dynamics of \factorsym{F2}, notice first that the decay
rate was set explicitly to 1. Thus, the total amount of \factorsym{F2}
in a time step is equal to the amount synthesized in that time step.

Regulation of \genesym{g2} by \factorsym{F1} is modelled by the
\codeword{activate} statement. The factor activating the gene is
mentioned before the keyword \codeword{activate}. Quantitatively,
activation is computed according to the Michaelis-Menten-equation. The
two parameters of \codeword{activate}, which are denoted by \aspec and
\amax, are analogous to the Michaelis-Menten parameter $K_M$ and
$v_{\mathrm{max}}$, respectively. Thus, as the concentration of
\factorsym{F1} approaches 2, the rate of synthesis of \factorsym{F2}
approaches $\frac{\amax \cdot \factorconc{F1}}{\aspec +
  \factorconc{F1}} = \frac{5 \cdot 2}{2 + 2} = 2.5$.

Fig.\ \ref{fig_twofactors} shows that synthesis of \factorsym{F2}
starts one time step after \factorsym{F1} begins to accumulate. This
is due to the implementation of \transsys{}: Factor synthesized in a
time step does not participate in gene regulation in that time step,
effects set in in the subsequent time step.


\subsubsection{Getting Advanced: Autoregulation}
\label{sect_cycler_final}

\begin{figure}
\centerline{
  \includegraphics[width=0.45\textwidth]{tr_ex06.eps}
  \includegraphics[width=0.45\textwidth]{tr_ex06_netgraph.eps}
}
\caption{\label{fig_cycler}
  Oscillatory dynamics in a regulatory network of two genes (left) and
  a graph representation of the regulatory network (right).
}
\end{figure}

Having seen how one gene can be regulated by a factor encoded by
another gene, the stage is now ready for presenting a \transsys{}
program that deserves being called a regulatory network:
\verbatimtabinput[8]{tr_ex06.tra}
Fig.\ \ref{fig_cycler} shows the temporal dynamics of the two factors
\factorsym{A}, which activates both genes and \factorsym{R}, which
represses both genes. Initially, \genesym{agene} is slightly activated
due to the \codeword{constitutive} statement in the
promoter. Accumulation of \factorsym{A} amplifies activation of
\genesym{agene}, but also results in activation of \genesym{rgene},
which encodes \factorsym{R}. The parameters of activation and
repression are chosen such that the repression effects of
\factorsym{R} eventually lead to a temporary shutdown of expression of
both genes. After that, decay of \factorsym{R} finally allows
constitutive expression of \genesym{agene} to set in again, which
starts the next oscillation.


\subsubsection{Visualizing the Regulatory Network}

The \transsys{} program introduced above is surely simple enough to be
understandable by just reading the code. Larger networks, however,
demand larger and more complex \transsys{} programs. The program
\prgname{transps} can be used to graphically render the regulatory
network structures encoded in a \transsys{} program:
\begin{quote}
\cmdline{transps example.tra example.ps}
\end{quote}
For the simple autoregulatory network shown above, the corresponding
graph is shown in Fig.\ \ref{fig_cycler}.


\subsection{Writing a \ltranssys{} Program}

\subsubsection{The Beginning}

\ltranssys{} uses the same principles as
\transsys{}, in particular, the concept of named blocks was applied in
\ltranssys{} whereever it seemed reasonable. Thus, an empty
\ltranssys{} looks like this:
\verbatimtabinput[8]{trl_ex00.trl}


\subsubsection{Defining a Symbols}

Symbols (from which strings are assembled) are the basic unit of
L-systems. In \ltranssys{} symbols have to be defined before they can
be used. So, let's define a symbol:
\verbatimtabinput[8]{trl_ex01.trl}


\subsubsection{Defining a Rule}

Rules are the centerpiece of traditional L-systems, and in
\ltranssys{}, they are mechanism which links realization of
morphogenetic processes to gene expression. But we'll postpone
introduction of \transsys{} programs with \ltranssys{} a little in
order to first demonstrate the basics of rule definition. A simple
example rule reads:
\verbatimtabinput{trl_ex02.trl}
As you might guess (at least, if you have some experience with
L-systems) this rule replaces each occurrence of a
\codeword{shoot\_piece} symbol with a string of two
\codeword{shoot\_piece} symbols. Repeated application yields a string
of \codeword{shoot\_piece} symbols which exponentially grows in length.


\subsubsection{Trying it out}

To actually see the rapidly growing string, it is necessary to provide
the system with an initial string to start with. This initial string
is called the axiom in L-system terminology, therefore, \ltranssys{}
uses a keyword called \codeword{axiom} for this purpose:
\verbatimtabinput[8]{trl_ex03.trl}
This \ltranssys{} program is certainly minimalistic, and it's not
original at all as it doesn't use anything \transsys{} specific, but
let's use it to see \ltranssys{} at work nonetheless:
\begin{quote}
\cmdline{ltrcheck -n 5 example.trl}
\end{quote}
This command prints out the result of the first $5$ derivations of the
axiom. The output is not really interesting, consisting just of lots
of \codeword{shoot\_piece}s, so it's not shown here.


\subsubsection{Adding Graphics}

The output not shown above is not just boring because the \ltranssys{}
program comprises just one symbol. String dumps are useful for
debugging, but the real strength of L-systems can only be seen if the
string is rendered graphically. Now, since \ltranssys{} lets us define
arbitrary symbols, it cannot know how what graphics we want to
associate with our symbols. Thus, we end up having to specify that too:
\verbatimtabinput[8]{trl_ex04.trl}
By adding this piece to the \ltranssys{} program developed so far, we
obtain a minimalistic system which can be used for 3D rendering with
the program \prgname{ltransgl}:
\begin{quote}
\cmdline{ltransgl example.trl}
\end{quote}
Running this program brings up an OpenGL window in which a grey
cylinder is displayed. The object can be moved along the X axis by
dragging with the left mouse button. Translation along the Y and Z
axes are possible by pressing and holding down the shift or control
key, respectively. The object can also be rotated around all axes by
dragging with the right mouse button, again using the shift or control
key to select the Y or Z axis. Pressing \keyboardin{n} computes shows
the next derivation step, \keyboardin{p} shows the preceding step. The
\keyboardin{Home} and \keyboardin{End} keys move to the initial step
(i.e.\ the axiom) and the last step computed so far, respectively.


\subsubsection{Changing Direction}

\begin{figure}
\centerline{
  \includegraphics[width=0.45\textwidth]{trl_ex05_glgraph.eps}
}
\caption{\label{fig_wiggle}
  Demo of \codeword{turn()} graphics command.
}
\end{figure}

Shoot pieces forming a straight chain seem somewhat unflexible. Of
course, \ltranssys{} allows changes in orientation. There are three
axes around which one can rotate in 3D, and rotations are done with
the graphics functions \codeword{turn()}, \codeword{roll()} and
\codeword{bank()}. Let's demonstrate the \codeword{turn()} function
with an example:
\verbatimtabinput[8]{trl_ex05.trl}
In this example, we have introduced two new symbols, \codeword{left}
and \codeword{right}, which are associated with turns to the left and
to the right, respectively. The rule was also modified to actually use
the new symbols. The graphical result is shown in Fig.\ \ref{fig_wiggle}.
You may be surprised about the irregular wiggling structure; it is due
to the fact that \codeword{left} and \codeword{right} symbols occur in
multiple repeats, sometimes cancelling each other out. If you're puzzled,
the \prgname{ltrcheck} program might help you.


\subsubsection{Push and Pop Symbols}

\begin{figure}
\centerline{
  \includegraphics[width=0.45\textwidth]{trl_ex06_glgraph.eps}
}
\caption{\label{fig_branch}
  A branched structure generated with \ltranssys{}.
}
\end{figure}

Drawing branched structures requires functions for saving the current
position and orientation, along with other information collecively
called a state, and for returning to the saved state at some later
time. These operations are called pushing (the state) and popping (the
state), respectively. In \ltranssys{}, pushing and popping are
graphical functions, so in order to use these operations, we firstly
have to define symbols for them and secondly, we have to associate
them with the corresponding graphics primitives. Here's a code
fragment that does all that:
\verbatimtabinput[8]{trl_ex06.trl}
This code shows that the square brackets, \codeword{[} and
\codeword{]}, can be used as symbols. The intention is to use them as
push and pop symbols, as shown here. In addition to introducing the
new symbols and their graphics, the rule has again been modified to
show off the new stuff. Fig.\ \ref{fig_branch} shows that the system
indeed yields a branched structure.


\subsubsection{Integrating \transsys{}}

\transsys{} is integrated into \ltranssys{} by extending the concept
of parametric L-systems: Symbols can be defined to have a \transsys{}
instance associated with them, as in
\verbatimtabinput[8]{trl_ex07.trl}
As you see, attaching a \transsys{} instance to a symbol is really
easy, the difficult part is to write the \transsys{} program in the
first place. For our example, we'll use the \codeword{cycler} program
listed in section \ref{sect_cycler_final}.


\subsubsection{Handling \transsys{} Instances in Rules}

Having defined that \codeword{meristem} symbols should have a
\codeword{cycler} instance attached, we now need a way to access
what's going on in the \codeword{cycler} instance and to use this
information for controlling the growth process modelled by
\ltranssys{} rules. Here's the code:
\verbatimtabinput[8]{trl_ex08.trl}
This example introduces several new constructs. Firstly, we see that a
\transsys{} instance associated with a symbol is given a local name
within the lefthand side of a rule. Here, the \transsys{} instance
associated with the \codeword{meristem} symbol is labelled
\codeword{t}.

Secondly, we see a new element in this rule: A condition which must be
fulfilled in order to activate the rule. In our example, the rule is
only activated if the concentration of factor \factorsym{A} in the
\transsys{} instance labelled \codeword{t} is greater than $0.91$.

Thirdly, the \transsys{} instance is also used for creating new
symbols on the righthand side of a rule. All \codeword{meristem}
symbols specified there are qualified with \codeword{(transsys~t:~)}.
This means that the factor concentration values from the \transsys{}
instance labelled \codeword{t} are copied into the \transsys{}
instances associated with the newly created symbols.


\subsubsection{A Plant Branching Under the Control of Genes}

\begin{figure}
\centerline{
  \includegraphics[width=0.45\textwidth]{trl_ex09_glgraph.eps}
}
\caption{\label{fig_transsysgrow}
  Another branched structure generated with \ltranssys{}. In this
  case, branching is controlled by gene activity within \transsys{}
  instances in the meristems which are graphically rendered as green
  spheres.
}
\end{figure}

Having introduced the handling of \transsys{} instances in rules, we
are finally ready to write an \ltranssys{} program in which branching
is controlled by gene activity within meristems. We just have to add
the graphics instructions for rendering \codeword{meristem} symbols:
\verbatimtabinput{trl_ex09.trl}
A graphical display of the virtual plant obtained with this
\ltranssys{} program is shown in Fig.\ \ref{fig_transsysgrow}.


\subsubsection{Visualizing Factor Concentrations}

In the \ltranssys{} program which we've developed now, factor
concentrations are causal for branching. So, it would be nice if we
could somehow see the oscillations in concentrations of factor
\factorsym{A} as our plant develops. This is easily feasible, as
factor concentrations are accessible within graphics definitions. All
we need is to modify the meristem graphics:
\verbatimtabinput{trl_ex10.trl}
This modification does not alter the morphology of the plant. But
instead of showing all meristems as spheres with a bright green
colour, the spheres are now coloured according to the concentrations
of the two factors. This is best seen in an animation. Such an
animation can be seen by running \prgname{ltransgl}: Firstly, let
\prgname{ltransgl} do a few hundred derivations. Here, the fact that
pressing \keyboardin{N} performs 50 derivations at once is handy. Once
you've done that, press \keyboardin{Home} to get back to the
axiom. Now press \keyboardin{>} to see an animation running through
all steps. A reverse animation can be shown by pressing \keyboardin{<}.


\section{Formal \transsys{} Specification}

\subsection{Constitutents of a \transsys{} program: Lexical Analysis}

The \transsys{} language consists of tokens (lexical elements), much
as many common computer languages, such as C. The token classes are:

\begin{itemize}

\item Comment: All characters between a \verb|#| (hash character) and
  the end of a line are considered a comment and are ignored by
  \transsys{}.

\item Number: All numbers in \transsys{} are real valued and are
  written in the usual notation, e.g.\ \verb|12|, \verb|3.14|,
  \verb|47.11e22| etc.
  
\item Identifier: An identifier is a string in which the first
  character is an alphabetical character and the subsequent characters
  are alphanumerical characters. The underscore \verb|_| is considered
  an alphabetical character. Identifiers are case sensitive. Their
  main purpose in \transsys{} is denoting genes and factors.
  
\item Keyword: The keywords reserved by the core \transsys{} language
  are \verb|activate|, \verb|constitutive|, \verb|decay|, \verb|default|,
  \verb|diffusibility|, \verb|factor|, \verb|gauss|, \verb|gene|,
  \verb|product|, \verb|promoter|, \verb|random|, \verb|repress|,
  \verb|transsys|.
  
  Additionally, the following keywords are reserved by \ltranssys{}:
  \verb|axiom|, \verb|bank|, \verb|box|, \verb|color|, \verb|cylinder|,
  \verb|graphics|, \verb|lsys|, \verb|move|, \verb|pop|, \verb|push|,
  \verb|roll|, \verb|rule|, \verb|sphere|, \verb|symbol|, \verb|turn|.

  Reserved keywords may only be used in the ways defined by the
  language specifications, i.e.\ they cannot be used as identifiers or
  for other purposes. Please see section \ref{future_plans} below for
  additional info and advice.

\item Operator: \transsys{} uses the following operators, which should
  look familiar to those programming in C or C++: \verb|<=|, \verb|>=|,
  \verb|==|, \verb|!=|, \verb|&&|, \verb|x|, \verb|+|, \verb|-|,
  \verb|*|, \verb|/|, \verb|!|, \verb|<|, \verb|>|, \verb|=|.
  
  Additionally, \ltranssys{} reserves and uses the production operator
  \verb|-->|.
  
\item Punctuation and Structure: Curly braces \verb|{|, \verb|}| are
  used to separate a \transsys{} specification into modular
  blocks. Parentheses \verb|(| and \verb|)| are used for explicitly
  specifying precedence in arithmetic and logical
  expressions. Individual statements are separated by semicolons
  \verb|;|, assignment lists in \ltranssys{} are separated by commas.

\end{itemize}


\subsection{Overall Structure}

A \transsys{} is specified by the keyword \keyword{transsys}, followed
by a name and a body, consisting of \transsys{} elements enclosed in
curly braces. \transsys{} elements are factor definitions and gene
definitions.

An empty \transsys{}, i.e.\ one with no factors or genes, is formally
allowed but it may be of limited use.

\begin{footnotesize}
\begin{verbatim}
transsys -> "transsys" identifier "{" transsys_element_list "}"
transsys_element_list -> /* empty */
transsys_element_list -> transsys_element_list factor_definition
transsys_element_list -> transsys_element_list gene_definition
\end{verbatim}
\end{footnotesize}


\subsection{Factor Definition}

A factor in \transsys{} is specified by the keyword \keyword{factor},
followed by a name and a body. Within the body, the factor's decay
rate and diffusion rate are specified. These specifications may be
omitted, in which case the default value of 0 is assumed for both
parameters.

Formally, the grammar allows multiple specifications of both
parameters; in the current parser implementation, the last
specification will become effective. However, relying on this
"feature" is strongly discouraged.
\begin{footnotesize}
\begin{verbatim}
factor_definition -> "factor" identifier "{" factordef_components "}"
factordef_components -> /* empty */
factordef_components -> factordef_components factordef_component
factordef_component -> "decay" ":" expr ";"
factordef_component -> "diffusibility" ":" expr ";"
\end{verbatim}
\end{footnotesize}


\subsection{Gene Definition}

A gene in \transsys{} is specified by the keyword \keyword{gene},
followed by a name and a body. The contents of the body are subdivided
into a promoter component and a product component. The promoter
component specifies the level of the gene's expression activity as a
function of the factor concentrations. The product component specifies
which factor is encoded by the gene (i.e.\ which factor is synthesized
upon the gene's expression).
\begin{footnotesize}ene's activation).
\begin{verbatim}
gene_definition -> "gene" identifier "{" promoter_component product_component "}"
\end{verbatim}
\end{footnotesize}


\subsubsection{Promoter Definition}

The promoter component specifies computation of the level of the
gene's expression as a function of the factor concentrations.
Currently, there are three types of statements possible in the
\keyword{promoter} component. Each statement is evaluated to compute a
contribution of activation (or repression). The activation contributed
by statement $i$ is denoted by $a_i$.

\keyword{constitutive} is the most generic type, this statement
specifies an expression determining an amount of activation:
\begin{equation}
a_i = \mbox{ result of evaluating expression}
\end{equation}
\keyword{activate} and \keyword{repress} statements are both
preceded by a factor name $\factorvar{f}$, and both have a list of two
expressions as arguments. The arguments determine the specificity,
denoted by $\aspec$, and the maximal rate of activation, denoted by
$\amax$. The actual amount of activation is calculated according to
the Michaelis-Menten-equation:
\begin{equation}
a_i = \frac{\amax \factorconc{\factorvar{f}}}{\aspec + \factorconc{\factorvar{f}}}
\end{equation}
Repression is calculated by the same formula with the sign reversed:
\begin{equation}
a_i = -\frac{\amax \factorconc{\factorvar{f}}}{\aspec + \factorconc{\factorvar{f}}}
\end{equation}
Both parameters $\aspec$ and $\amax$ are specified by expressions,
which allows modelling of modulation of activation by protein-protein
interactions. The amount of product $\factorvar{p}$ synthesized
through expression of gene $\genevar{g}$ in a time step is given by
\begin{equation}
\label{eq_delta_g_fconc}
\Delta_{\genevar{g}} \factorconc{\factorvar{p}} = \left\{
\begin{array}{ll}
a_{\mathrm{total}} := \sum_i a_i & \mbox{ if } a_{\mathrm{total}} > 0 \\
0 & \mbox{ otherwise} \\
\end{array}
\right.
\end{equation}
\begin{footnotesize}
\begin{verbatim}
promoter_component -> "promoter" "{" promoter_statements "}"
promoter_statements -> promoter_statement
promoter_statements -> promoter_statements promoter_statement
promoter_statement -> "constitutive" ":" expr ";"
promoter_statement -> factor_combination ":" "activate" "(" expr "," expr ")" ";"
promoter_statement -> factor_combination ":" "repress" "(" expr "," expr ")" ";"
factor_combination -> identifier
factor_combination -> factor_combination "+" identifier
\end{verbatim}
\end{footnotesize}


\subsubsection{Gene Product Definition}

\begin{footnotesize}
\begin{verbatim}
product_component -> "product" "{" product_statements "}"
product_statements -> "default" ":" identifier ";"
\end{verbatim}
\end{footnotesize}


\subsection{Expressions}

\begin{footnotesize}
\begin{verbatim}
expr -> expr LOGICAL_OR and_expr
expr -> and_expr
and_expr -> and_expr LOGICAL_AND not_expr
and_expr -> not_expr
not_expr -> '!' not_expr
not_expr -> cmp_expr
cmp_expr -> cmp_expr '<' arithmetic_expr
cmp_expr -> cmp_expr '>' arithmetic_expr
cmp_expr -> cmp_expr LOWER_EQUAL arithmetic_expr
cmp_expr -> cmp_expr GREATER_EQUAL arithmetic_expr
cmp_expr -> cmp_expr EQUAL arithmetic_expr
cmp_expr -> cmp_expr UNEQUAL arithmetic_expr
cmp_expr -> arithmetic_expr
arithmetic_expr -> arithmetic_expr '+' term
arithmetic_expr -> arithmetic_expr '-' term
arithmetic_expr -> term
term -> term '*' value
term -> term '/' value
term -> value
value -> REALVALUE
value -> '(' expr ')'
value -> RANDOM '(' expr ',' expr ')'
value -> GAUSS '(' expr ',' expr ')'
value -> IDENTIFIER
value -> IDENTIFIER '.' IDENTIFIER
\end{verbatim}
\end{footnotesize}


\subsection{\ltranssys{}}

\begin{footnotesize}
\begin{verbatim}
lsys -> LSYS_DEF IDENTIFIER @1 '{' lsys_element_list '}'
lsys_element_list -> lsys_element
lsys_element_list -> lsys_element_list lsys_element
lsys_element -> symbol_definition
lsys_element -> axiom_definition
lsys_element -> rule_definition
lsys_element -> graphics_definition
symbol_definition -> SYMBOL_DEF IDENTIFIER ';'
symbol_definition -> SYMBOL_DEF IDENTIFIER '(' IDENTIFIER ')' ';'
symbol_definition -> SYMBOL_DEF '[' ';'
symbol_definition -> SYMBOL_DEF '[' '(' IDENTIFIER ')' ';'
symbol_definition -> SYMBOL_DEF ']' ';'
symbol_definition -> SYMBOL_DEF ']' '(' IDENTIFIER ')' ';'
axiom_definition -> AXIOM_DEF production_element_string ';'
rule_definition -> RULE_DEF IDENTIFIER '{' rule_components '}'
rule_components -> rule_lhs ':' expr ARROW rule_rhs
rule_components -> rule_lhs ARROW rule_rhs
rule_lhs -> lhs_element_string
lhs_element_string ->           /* empty */
lhs_element_string -> lhs_element_string lhs_element
lhs_element -> IDENTIFIER
lhs_element -> IDENTIFIER '(' IDENTIFIER ')'
rule_rhs -> production_element_string
production_element_string ->            /* empty */
production_element_string -> production_element_string production_element
production_element -> IDENTIFIER
production_element -> IDENTIFIER '(' transsys_initializer ')'
production_element -> '['
production_element -> ']'
transsys_initializer -> source_transsys_specifier assignment_list
transsys_initializer -> assignment_list
source_transsys_specifier -> TRANSSYS_DEF IDENTIFIER ':'
assignment_list ->              /* empty */
assignment_list -> assignment
assignment_list -> assignment_list ',' assignment
assignment -> IDENTIFIER '=' expr
graphics_definition -> GRAPHICS_DEF '{' symgraph_list '}'
symgraph_list ->                /* empty */
symgraph_list -> symgraph_list symgraph
symgraph -> IDENTIFIER '{' graphcmd_list '}'
symgraph -> '[' '{' graphcmd_list '}'
symgraph -> ']' '{' graphcmd_list '}'
graphcmd_list ->                /* empty */
graphcmd_list -> graphcmd_list graphcmd
graphcmd -> MOVE '(' expr ')' ';'
graphcmd -> PUSH '(' ')' ';'
graphcmd -> POP '(' ')' ';'
graphcmd -> TURN '(' expr ')' ';'
graphcmd -> ROLL '(' expr ')' ';'
graphcmd -> BANK '(' expr ')' ';'
graphcmd -> SPHERE '(' expr ')' ';'
graphcmd -> CYLINDER '(' expr ',' expr ')' ';'
graphcmd -> BOX '(' expr ',' expr ',' expr ')' ';'
graphcmd -> COLOR '(' expr ',' expr ',' expr ')' ';'
\end{verbatim}
\end{footnotesize}


\subsection{Future Development Plans}
\label{future_plans}

\subsubsection{Additional Keywords}

It is a notorious problem that the set of keywords of a computer
language increases as the language evolves, and that the introduction
of new keywords may result in old code becoming malfunctional and
"illegal".

Extending \transsys{}, particularly by integration of additional
modelling methods, is definitely intended. However, it is not clear
which keywords will be introduced in this process. Therefore, at this
time, only some rather general design principles can be given. These
may provide some guidance for the cautiously minded:
\begin{itemize}
\item No keywords containing any capital letters will be introduced.
\item No keywords beginning with an underscore will be introduced, and
  introduction of keywords containing underscores is unlikely.
\item Introduction of keywords containing numbers is also unlikely
\item Most likely candidates for keywords are on the one hand words
  that are used as keywords in other computer languages, an on the
  other hand terms for generic biological structures. Examples for the
  latter category are \keyword{cell}, \keyword{tissue},
  \keyword{organ}, \keyword{enhancer}, \keyword{chromosome} etc. It is
  recommended to avoid such terms in order to minimize troubles due to
  future development of \transsys{}.
\end{itemize}


\end{document}

